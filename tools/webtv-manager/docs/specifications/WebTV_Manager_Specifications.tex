\documentclass[10pt,a4paper]{article}

\usepackage[utf8x]{inputenc}
\usepackage[T1]{fontenc}
\usepackage[french]{babel}
\usepackage{lmodern}

\usepackage{hyperref}
\usepackage{graphicx}
\usepackage{array}
\usepackage{url}

\author{Eric Beuque}
\title{Spécification du projet de développement d'un site Web de gestion de WebTV}
\date{\today}

\begin{document}

\maketitle

\section*{Introduction}

Dans le cadre du projet FreetuxTV, on souhaite développer une application Web capable de gérer les WebTV. Ce document explique quels sont les objectifs que l'application doit atteindre, les contraintes techniques et les quelques briques fondamentales pour mener à bien ce projet.

\section*{Révisions}

\begin{itemize}
\item 31 aout 2010 : Mise au clair de chaque partie de la spécification.
\item 5 novembre 2009 : Première version.
\end{itemize}

\section{Objectifs}

FreetuxTV possède une base de donnée de WebTV et WebRadio géré statiquement par fichier M3U. L'objectif principal est de fournir une interface Web pour que les utilitateurs de FreetuxTV puisse facilement transmettre à l'équipe des liens de WebTV ou WebRadio à ajouter dans les playlist. Ces liens seront soumis à une modération. L'interface permettra aussi à un utilisateur de signaler que des liens sont morts. A partir des liens enregistrés dans la base de donnée, on pourra facilement créer des playlists dynamiques suivant certains critères accessible simplement via une URL.

Voici le détail des fonctionnalités non exhaustive que l'application se devra de remplir. Bien entendu, toutes nouvelles idées pouvant améliorer les propositions citées ci-dessous sont les bienvenues et peuvent être transmises à l'équipe de FreetuxTV.

\subsection{Gestion des utilisateurs}

L'application devra avoir un système de gestion des utilisateurs. Un utilisateur possède un nom d'utilisateur, un mot de passe et une adresse E-Mail.
Ceux ci pourront aussi posséder une liste de droits (liste réduites mais pouvant évolué par la suite) :

\begin{itemize}
\item Droit d'édition de chaîne : Une base de donnée de chaîne et radio sera crée en interne, ce droit permet de pouvoir en ajouter et les modifier.
\item Droit de modération des flux : Pour des problèmes de droits, les flux proposés devront être modérer afin que l'application ne devienne pas illégal en proposant des contenus abusifs. Ce droit permet à un utilisateur de valider une URL et de la rendre disponible dans les playlists générés dynamiquement.
\item Droit d'administration : Ce droit permet à un utilisateur d'affecter des droits à d'autres utilisateurs.
\end{itemize}

Tout les utilisateurs pourront soumettre une URL à inclure dans les playlists. Il pourrons aussi signaler les WebTV dont le lien est mort et proposé des nouveaux flux.

\subsection{Gestion des WebTV}

Il est important de différencier les chaînes proprement dîtes et les flux des chaînes. En effet, gardons à l'esprit qu'une chaîne possède plusieurs moyens de diffusion et peut donc posséder plusieurs flux vidéos. Par exemple, France 3 reçu avec l'opérateur Free et Neuf sont la même chaîne mais pas le même flux. De ce fait, les informations d'un flux doivent être distinct des informations de la chaîne.

\subsubsection{Soumission d'un flux}

Lorsqu'un utilisateur (connecté ou non) veut soumettre un flux, on lui demandera le nom du flux, l'URL et éventuellement les options VLC (stocké dans un champs texte) pouvant faire fonctionner le flux. Un champs permettra aussi de spécifié le FAI requis pour lire le flux. Il sera aussi possible d'ajouter directement une remarque sur le flux (par exemple pour spécifier que le flux n'est disponible que pour un pays ou une plage horaire).

On doit stocker la date de soumission du flux et l'utilisateur l'ayant soumis. Si un utilisateur n'est pas enregistré on lui demandera de saisir en plus un pseudo et une adresse E-Mail.

Un flux possède aussi un cycle de vie : proposé, validé, invalide, mort.

Seul modérateur aura la possibilité de changer le statut du lien. Il pourra aussi en modifier les informations enregistrées (Nom, URL...).

Il serait une bonne chose de pouvoir gérer l'ajout de commentaire de la part des utilisateurs pour un flux.

\subsubsection{Rattachement à une chaîne connu}

Un flux pourra être rattaché à une chaîne TV. Ces chaînes comportent de nombreux attributs : un nom, une langue, un pays d'origine, un logo, un type de chaînes (généraliste, sport, cinéma...), un site web, un lien direct vers le programme tv, une ou plusieurs page wikipedia suivant la langue... Certains de ces attributs seront utilisé pour rechercher un flux via un formulaire de recherche.

Seul un utilisateur ayant le droit d'édition de chaînes pourra ajouter ou modifier des chaînes dans la base de donnée. Un modérateur pourra au moment de la validation du flux choisir la chaîne à rattaché au flux.

\subsubsection{Recherche de chaîne}

Un formulaire de recherche permettra la recherche des flux ou des chaînes en fonction de leur nom, leur type, leur langue, leur pays ou leur FAI.

\subsection{Génération de playlist}

Le but de s'amuser à faire une base de donnée des WebTV est de pouvoir facilement générer des playlist m3u en fonction de divers paramètres. Par exemple, avoir toutes les chaînes de sport en français. On doit pouvoir générer la playlist uniquement avec des paramètres en URL (ex : \url{http://webtv/playlist/?type=sport&lang=fr}).

Cette URL sera généré automatiquement à partir du formulaire de recherche et affiché au niveau des résultats. Seul les flux ayant un statut valide seront généré.

\subsection{Web Service}

Cette fonction n'est pas une priorité dans l'immédiat, mais il faut prévoir qu'un jour elle pourrait être indispensable. Un Web Service pourrait permettre à FreetuxTV de plus facilement dialoguer avec la base de donnée.

\subsubsection{Détection de lien mort}

On pourrait utiliser le WebService pour que FreetuxTV communique à la base de donnée lorsqu'un lien d'une WebTV est mort afin que les playlist soit automatiquement purger, ceci devrait être toutefois coupler à un système de modération.

\subsubsection{Mesure d'audience}

On pourrait aussi utiliser le WebService afin d'établir un système de mesure d'audience, afin de connaître les chaînes les plus regarder avec FreetuxTV.

\section{Base de donnée}

Voici un premier jet de l'architecture de la base de donnée.

\subsection{Table wtvm\_Lang}

Cette table contient les attributs concernant une langue. Les tables seront toutes préfixées par wtvm\_.

\begin{itemize}
\item code : code ISO\_639-1 de la langue (\url{http://fr.wikipedia.org/wiki/Liste\_des\_codes\_ISO\_639-2})
\item label : nom anglais de la langue
\end{itemize}

\subsection{Table wtvm\_Country}

Cette table contient les attributs concernant un pays.

\begin{itemize}
\item code : code ISO\_3166-1 du pays (\url{http://fr.wikipedia.org/wiki/ISO\_3166-1})
\item label : nom anglais du pays
\end{itemize}

\subsection{Table wtvm\_TVChannelType}

Cette table contient les attributs concernant le type d'un chaîne.

\begin{itemize}
\item id : identifiant unique
\item label : nom anglais du type de chaîne
\end{itemize}

\subsection{Table wtvm\_TVChannel}

Cette table contient les attributs concernant une chaîne.

\begin{itemize}
\item id : identifiant unique
\item name : nom officiel de la chaîne
\item lang\_id : identifiant de la langue principal de la chaîne
\item country\_code : identifiant du pays de la chaîne
\item channeltype\_id : identifiant du type de la chaîne
\item logo : nom du logo stocker sur le serveur
\item website : url du site web officiel de la chaîne
\item guideurl : url direct du guide officiel de la chaîne
\end{itemize}

\subsection{Table wtvm\_TVChannelStream}

Cette table contient les attributs concernant le flux d'une chaîne TV.

\begin{itemize}
\item id : identifiant unique
\item name : nom donné au flux
\item url : url du flux
\item required\_isp : nom du FAI requis
\item comments : commentaire éventuel sur le flux
\item tvchannel\_id : identifiant de la chaîne associé
\end{itemize}

\subsection{Table wtvm\_User}

Cette table contient les attributs concernant un utilisateur de l'application.

\begin{itemize}
\item id : identifiant unique
\item login : nom d'utilisateur
\item password : mot de passe crypté
\item email : e-mail de l'utilisateur
\item rights : liste des droits de l'utilisateur (flags)
\end{itemize}

\section{Contraintes techniques}

Le projet devra remplir quelques contraintes techniques afin de pouvoir facilement le faire évolués.

\begin{itemize}
\item Toutes les parties du code (nom des variable, fonctions, commentaires...) devront être codés en anglais afin qu'un éventuel développeur étranger puisse facilement intégrer le projet.
\item L'application devra être codé en PHP/MySQL.
\item L'utilisation d'un framework tel que CakePHP ou Pluf serait bénéfique pour le projet (internationalisation, MVC...).
\end{itemize}


\end{document}
